
\documentclass[12pt,]{scrartcl}
\usepackage{lmodern}
\usepackage{amssymb,amsmath}
\usepackage{ifxetex,ifluatex}
\usepackage{fixltx2e} % provides \textsubscript

% Remove space around section headings
\usepackage[compact]{titlesec}
\titlespacing{\section}{0pt}{2ex}{1ex}
\titlespacing{\subsection}{0pt}{1ex}{0ex}
\titlespacing{\subsubsection}{0pt}{1ex}{0ex}

\ifnum 0\ifxetex 1\fi\ifluatex 1\fi=0 % if pdftex
  \usepackage[T1]{fontenc}
  \usepackage[utf8]{inputenc}
\else % if luatex or xelatex
  \ifxetex
    \usepackage{mathspec}
  \else
    \usepackage{fontspec}
  \fi
  \defaultfontfeatures{Ligatures=TeX,Scale=MatchLowercase}
\fi

% use upquote if available, for straight quotes in verbatim environments
\IfFileExists{upquote.sty}{\usepackage{upquote}}{}

% use microtype if available
\IfFileExists{microtype.sty}{%
\usepackage[]{microtype}
\UseMicrotypeSet[protrusion]{basicmath} % disable protrusion for tt fonts
}{}

\PassOptionsToPackage{hyphens}{url} % url is loaded by hyperref

\usepackage[unicode=true]{hyperref}
\hypersetup{pdfborder={0 0 0}, breaklinks=true}

\urlstyle{same}  % don't use monospace font for urls

\usepackage[top=1.0in,bottom=0.9in,left=1.15in,right=1.15in]{geometry}

\IfFileExists{parskip.sty}{%
\usepackage{parskip}
}{% else
\setlength{\parindent}{0pt}
\setlength{\parskip}{6pt plus 2pt minus 1pt}
}

\setlength{\emergencystretch}{3em}  % prevent overfull lines
\providecommand{\tightlist}{%
  \setlength{\itemsep}{0pt}\setlength{\parskip}{0pt}}
\setcounter{secnumdepth}{0}

% Redefines (sub)paragraphs to behave more like sections
\ifx\paragraph\undefined\else
\let\oldparagraph\paragraph
\renewcommand{\paragraph}[1]{\oldparagraph{#1}\mbox{}}
\fi
\ifx\subparagraph\undefined\else
\let\oldsubparagraph\subparagraph
\renewcommand{\subparagraph}[1]{\oldsubparagraph{#1}\mbox{}}
\fi

% set default figure placement to htbp
\makeatletter
\def\fps@figure{htbp}
\makeatother

\usepackage{hyperref}
\usepackage{multirow}
\usepackage{xcolor}
\usepackage{float} % here for H placement parameter

\definecolor{cvlinkcolor}{RGB}{6,125,233}
\hypersetup{colorlinks,linkcolor=cvlinkcolor,citecolor=cvlinkcolor,urlcolor=cvlinkcolor,bookmarksnumbered,bookmarks=false,pdfinfo={Title={Curriculum Vitae},Author={Fernando Paolo},Creator={Fernando Paolo},Subject={Curriculum Vitae},Keywords={Curriculum Vitae},CreationDate={D:20180702}}}
\renewcommand\labelenumi{[\theenumi]}

%----------------------------------------------------------------------------------------

\date{}

\begin{document}

\begin{table}[h]
{\def\arraystretch{1.1}\tabcolsep=0pt
\begin{tabular}{p{0.60\linewidth}p{0.05\linewidth}p{0.35\linewidth}}

\multirow{1}{*}{\LARGE \textbf{Salem Alqahtani}} &  &  \\

& & \\

%Jet Propulsion Laboratory & \multicolumn{1}{r}{Phone:\;\;} & \multicolumn{1}{l}{$+$1 858-752-1106} \\

State University of New York at Buffalo & \multicolumn{1}{r}{Email:\;\;} &\multicolumn{1}{l}{\href{salemmoh@buffalo.edu}{salemmoh@buffalo.edu}} \\

1185 Youngs Rd, Amherst NY, 14221, USA. & \multicolumn{1}{r}{Website:\;\;} & \multicolumn{1}{l}{\url{www.salemmoh.com/main.html}} \\
  
\end{tabular}}
\end{table}

%----------------------------------------------------------------------------------------

\textbf{Research summary:} Currently, I am working on Blockchain consensus protocols to develop a new system consensus algorithm. Broadly, I am interesting in distributed systems. \\

%----------------------------------------------------------------------------------------

\subsection{Professional Preparation}

%Postdoc at NASA Jet Propulsion Laboratory, California Inst. of Technology, 2017--2019  \\
%Ph.D. in Computer Science, Scripps Oceanography, University of California, San Diego, 2015  \\ %#2010--2015
M.S. in Computer Science, University of Connecticut, Storrs, CT, 2015 \\  % 2007--2009
B.S. in Computer Science, King Khalid University, ABHA, KSA, 2010 \\ % 2002--2006

%----------------------------------------------------------------------------------------

%\subsection{Appointments}

%2017--2019 \, Postdoc Researcher at California Institute of Technology, NASA-JPL \\


%----------------------------------------------------------------------------------------

\subsection{Honors \& Awards}

2010 \, Second Honor Degree from King Khalid University \\


%----------------------------------------------------------------------------------------

\subsection{Teaching Experience}

2010 \, Data Structure , King Khalid University, KSA \\


%----------------------------------------------------------------------------------------

\subsection{Fieldwork Experience}

2014 \, Wireless Sensor Networks, UC-Berkeley, CA (3 days, one trips) \\

%----------------------------------------------------------------------------------------

\subsection{Publications}

%----------------------------------------------------------------------------------------

\subsubsection{Conference articles}

\begin{enumerate}
\leftskip-0.13in

\item Fricker H.A., \textbf{F.S. Paolo}, M.R. Siegfried, S. Adusumilli, Short-term changes in Antarctica’s ice shelves are key to predicting their long-term fate, \textit{The Conversation} (2018). \href{https://theconversation.com/short-term-changes-in-antarcticas-ice-shelves-are-key-to-predicting-their-long-term-fate-95207}{theconversation.com}



\end{enumerate}

%----------------------------------------------------------------------------------------

\subsubsection{Theses and dissertations}

\begin{enumerate}
\leftskip-0.13in % for numbers

\item \textbf{Paolo F.S.} (2015), \textit{Interannual and decadal variations of ice shelves using multi-mission satellite radar altimetry, and links with oceanic and atmospheric forcings}, PhD Dissertation, University of California, ProQuest Dissertations Publishing, 127 pages. \href{http://fspaolo.net/work/phd/}{fspaolo.net/work/phd}


\end{enumerate}


\subsection{Additional Training \& Certifications}

Cutting Edge Deep Learning for Coders (part 2), 2019 \\
\textit{Fast.ai, online course (20 hours of lessons)}

Pre-Antarctic Training for the Brazilian Antarctic Program, 2003 \\
\textit{Brazilian Navy, Rio de Janeiro (7 days at sea $+$ 7 days in the mountains)} \\

%----------------------------------------------------------------------------------------



%----------------------------------------------------------------------------------------

\subsection{Other Skills \& Qualifications}

\subsubsection{Programming}

\emph{See my GitHub at \url{https://github.com/fspaolo}}

Python (using since 2005) \, C/C++ \, Fortran 90/95 \, JavaScript \, R \\
Matlab \, Shell-Script \, SQL \, HTML \, CSS \, JSON \, YAML \, \LaTeX \, Git

\subsubsection{Development}

\emph{Lead developer of the following projects:}

\href{https://#}{CAP-Toolkit} -- NASA's JPL Cryosphere Altimetry Processing Toolkit {\small \ (private repo)} \\
\href{https://github.com/fspaolo/altimpy}{AltimPy} -- Set of tools for processing satellite altimetry data {\small \ (Python library)} \\



\subsubsection{Supercomputing}

\emph{Experience with the following systems:}

Triton cluster (4,000 cores, 100 Tflops) @ San Diego Supercomputer Center \\



\subsubsection{Languages}

\textit{English} (professional proficiency) \, \textit{Portuguese} (native speaker) \, \textit{Spanish} (native speaker) \\

%----------------------------------------------------------------------------------------

\subsection{Professional Service}

\subsubsection{Reviewer}

Nature, Nature Geoscience, Geophysical Research Letters, Earth and Planetary Science Letters, Marine Geodesy, Journal of Glaciology, The Cryosphere.

NASA: High Mountain Asia Panel, Washington DC, Jul 2016 \\
National Science Foundation (NSF): Polar Programs / Antarctic Research \\
UK Parliament: POST Note ``Understanding Rising Sea Levels'', \href{https://researchbriefings.parliament.uk/ResearchBriefing/Summary/POST-PN-0555}{www.parliament.uk}

\subsubsection{(Co-)Organizer}

2016 \, International Glaciological Society Symposium, La Jolla/CA, Jul 2016 \\
2012 \, Earth Section Seminar Series at Scripps Oceanography \\
2011 \, Geophysics Seminar Series at Scripps Oceanography \\
2002 \, 1st Brazilian Oceanography Symposium, S\~ao Paulo, Aug 2002

\subsubsection{Mission support}

Contributed to NASA's ICESat-2 Science Team

%\subsubsection{Webmaster}

%IGS Symposium 2011 website: \href{http://glaciology2.ucsd.edu/igs2011/}{glaciology2.ucsd.edu/igs2011}
%Scripps Glaciology Group website: \href{http://glaciology.ucsd.edu/}{glaciology.ucsd.edu}

\subsubsection{Affiliations}

American Geophysical Union (\href{http://sites.agu.org/}{AGU}) \, European Geosciences Union (\href{http://www.egu.eu/}{EGU}) \\

%----------------------------------------------------------------------------------------

\subsection{Conference Presentations}

(\emph{First author abstracts only})

\begin{enumerate}
\leftskip-0.13in % for numbers

\item Paolo et al. (July 2019), \textit{25 years of Antarctic ice shelf melt measured from space}, JPL 2019 Postdoc Research Day, Pasadena/California %poster

\item Paolo et al. (May 2019), \textit{Time-varying melt rates from Antarctic ice shelves}, JPL OSE Meeting, Pasadena/California %talk

\item Paolo et al. (2007), \textit{Caracteriza\c{c}\~ao da morfodin\^amica de fundo da Barra de Canan\'eia atrav\'es de m\'etodos geof\'isicos}, 14th Scientific Initiation Symposium of the University of S\~ao Paulo %poster % [1]

% [1] https://uspdigital.usp.br/siicusp/cdOnlineTrabalhoVisualizarResumo?numeroInscricaoTrabalho=3160&numeroEdicao=14

\end{enumerate}

%----------------------------------------------------------------------------------------

\subsection{Press}

\emph{Article URLs available at \url{http://fspaolo.net/work/press}}

\textbf{Interviews given to:} \textit{Los Angeles Times}, \textit{The Washington Post}, \textit{Reuters}, \textit{BBC News}, \textit{The Wall Street Journal}, \textit{The Guardian}, \textit{Neewsweek}, \textit{The Weather Channel}, \textit{CBS News TV}, \textit{Carbon Brief}, \textit{Science/AAAS News}, etc.

%----------------------------------------------------------------------------------------

\vfill{} % Whitespace before final footer

\begin{center}
{\scriptsize Last updated: \today\- •\- Compiled in \LaTeX}
\end{center}

\end{document}